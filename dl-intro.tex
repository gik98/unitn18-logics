\chapter{DL - Introduction}

La logica descrittiva è una famiglia di formalismi che permettono una rappresentazione della conoscenza in un dominio ("mondo"). DL prevede prima la definizione della \textbf{terminologia} rilevanti del dominio (classi di oggetti e gerarchia fra queste); questi concetti sono poi impiegati per specificare proprietà sugli \textbf{oggetti} che compongono il dominio (la "descrizione" del "mondo").

Obiettivo della DL è rispondere a task di ragionamento sulla struttura della terminologia (p.es.: il termine - la classe $Canary$ è una specializzazione di $Bird$?) oppure sugli oggetti che compongono il mondo (p.es.: l'oggetto $foo$ gode della proprietà - è membro della classe $Canary$?).

\subsection{Knowledge Base}

In DL, la Knoledge Base è composta dall'insieme della \textit{terminologia} e della \textit{descrizione} del mondo. Queste due componenti sono chiamate \textbf{TBox} e \textbf{ABox}.
\begin{enumerate}
\item Il \textbf{TBox} è l'insieme della \textit{terminologia}, il "vocabolario" della KB. Il vocabolario consiste di \textit{atomic concepts}, che denotano classi di oggetti, e \textit{roles}, che denotano relazioni fra \textit{concepts}. Ulteriori \textit{concepts} derivati possono essere composti combinando \textit{atomic roles}; equivalentemente per le \textit{roles}.
\item La \textbf{ABox} introduce \textit{oggetti} del mondo, assegnando loro un nome e effettuando \textit{asserzioni} su essi; le asserzioni sono espresse nella terminologia della TBox.
\end{enumerate}

\subsection{Attributive Concept Language with Complements}
$\mathcal{ALC}$ (Attributive Concept Language with Complements) è un linguaggio di description logics. La descrizione del mondo in $\mathcal{ALC}$ comprende \textit{concept atomici} e \textit{roles atomici} (proprietà che descrivono i concepts); descrizioni complesse possono essere ottenute componento descrizioni atomiche secondo le regole sintatiche.

\begin{fdefinition}[Linguaggio $\mathcal{ALC}$ per la TBox]
I \textit{concepts} sono composti secondo le seguenti regole sintattiche.

\textbf{Concept atomici:}
\begin{enumerate}
\item $A$ un concept atomico \\ Esempio: $Person$
\item $\top$ il concept universale
\item $\bot$ l'opposto del concept universale
\item $\lnot A$ la negazione di un concept atomico
\end{enumerate}
\textbf{Concept composti (well formed formulas):}
\begin{enumerate}
\item $F \coloneqq C \sqcap D$ l'intersezione fra due concept \\ Esempio: $Woman \coloneqq Person \sqcap Female$
\item $F \coloneqq C \sqcup D$ l'unione di due concept \\ Esempio: $Parent \coloneqq Father \sqcup Mother$
\item $F \coloneqq \forall R. C$ la restrizione universale di un concept ad un role \\ Esempio: $F \coloneqq Person \sqcap \forall hasChild. Female$ (F descrive le persone che hanno tutte le figlie femmina)\\ Esempio: $G \coloneqq Person \sqcap \forall hasChild.\bot$ (G descrive tutte le persone che non hanno figli)
\item $F \coloneqq \exists R. C$ la restrizione esistenziale di un concept rispetto ad un role \\ Esempio: $Parent \coloneqq Person \sqcap \exists hasChild.\top$ (Parent sono le persone che hanno (almeno) un figlio) \\ Esempio: $F \coloneqq Person \sqcap hasChild. Female$ (F descrive le persone che hanno (almeno) una figlia femmina)
\end{enumerate}
\end{fdefinition}
