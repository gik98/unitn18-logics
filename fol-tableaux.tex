\chapter{FOL - Tableaux}

Il tableaux in FOL funziona in modo simile a quello della PL, introduce due nuove regole di riduzione legate ai quantificatori e, a differenza della PL, può non terminare.

\subsection{Regole di riduzione}

\begin{tabular*}{\textwidth}{c @{\extracolsep{\fill}} c}
$\alpha$ rules & $\lnot \lnot$ elimination \\
\\
%alpha rules
\begin{tabular}{c c c}
\begin{tabular}{c}
$\phi \land \psi$ \\
\hline
$\phi$ \\
$\psi$ \\
\end{tabular} &
\begin{tabular}{c}
$\lnot (\phi \lor \psi)$ \\
\hline
$\lnot \phi$ \\
$\lnot \psi$ \\
\end{tabular} &
\begin{tabular}{c}
$\lnot (\phi \supset \psi)$ \\
\hline
$\phi$ \\
$\lnot \psi$ \\
\end{tabular}
\end{tabular} &
%lnot lnot elimination
\begin{tabular}{c}
$\lnot \lnot \phi$ \\
\hline
$\phi$
\end{tabular}\\
\\
$\beta$ rules & Branch closure \\
%beta rules
\begin{tabular}{c c c}
\begin{tabular}{c}
$\phi \lor \psi$ \\
\hline
$\phi$ \vline \hspace{1mm} $\psi$ 
\end{tabular} &
\begin{tabular}{c}
$\lnot (\phi \land \psi)$ \\
\hline
$\lnot \phi$ \vline \hspace{1mm} $\lnot \psi$ 
\end{tabular} &
\begin{tabular}{c}
$\phi \supset \psi$ \\
\hline
$\lnot \phi$ \vline \hspace{1mm} $\psi$ 
\end{tabular}
\end{tabular} &
%branch closure
\begin{tabular}{c}
$\phi$\\
$\lnot\phi$\\
\hline
$\mathrm{X}$
\end{tabular}\\
\\
$\gamma$ rules & $\delta$ rules\\
\begin{tabular}{c c}
%gamma rules
\begin{tabular}{c}
$\forall x . \varphi(x)$ \\
\hline
$\varphi(t)$
\end{tabular} &
\begin{tabular}{c}
$\lnot \exists x . \varphi(x)$ \\
\hline
$\lnot \varphi(t)$
\end{tabular}
\end{tabular} &
\begin{tabular}{c c}
%delta rules
\begin{tabular}{c}
$\lnot \forall x . \varphi(x)$ \\
\hline
$\lnot \varphi(c)$
\end{tabular} &
\begin{tabular}{c}
$\exists x . \varphi(x)$ \\
\hline
$\varphi(c)$
\end{tabular}
\end{tabular}\\
\\
\end{tabular*}

\begin{enumerate}
\item Nelle $\gamma$ rules, $t$ rappresenta una qualunque costante, anche già presente nel tableaux.
\item Nelle $\delta$ rules, $c$ rappresenta una \textit{nuova costante}, non presente nel tableaux.
\end{enumerate}

Perché funziona? Spiegazione:
\begin{enumerate}
\item $\forall x . \varphi(x)$ significa che $\varphi$ deve essere soddisfatta per ogni elemento del dominio di interpretazione $\Delta$. Una \textit{costante} $t$ è un elemento del dominio $\Delta$. Pertanto, $\forall x. \varphi(x) \supset \varphi[x/t]$.\\\textit{È lecito applicare più volte una $\gamma$ rule}, utilizzando costanti diverse: l'applicazione rimane valida infatti per ogni costante del dominio. \textit{Può essere necessario applicare più volte una $\gamma$ rule}, ad esempio, per cercare due branch closure su due rami diversi, con due variabili diverse in conflitto (e che condividono come ancestor un quantificatore universale).
\item $\exists x. \varphi(x)$ è equivalente a $\exists c \in \Delta | \varphi(c)$, dunque $\exists x. \varphi(x)$ è soddisfacibile sse $\varphi(c)$ è soddisfacibile.\\È importante che ogni $\delta$ rule introduca una nuova costante: riutilizzare la stessa nello sviluppo di esistenziali diversi modificherebbe la formula di partenza; imporrebbe infatti che \textit{lo stesso} elemento del dominio $\Delta$ verifichi \textit{ogni} esistenziale!
\end{enumerate}

\vspace{1em}

\subsection{Infinite tableaux}
Se il \textit{tableaux} è applicato ad una formula FOL $\varphi$, e questa formula è \textbf{insoddisfacibile}, la costruzione del tableaux \textbf{termina} in un numero finito di passi.

Se, però, il \textit{tableaux} è applicato ad una formula \textbf{soddisfacibile}, la sua costruzione \textbf{può terminare} in un numero finito di passi e lasciare qualche ramo aperto, \textbf{oppure non terminare}.

Un \textit{tableaux} nel quale compare un \textit{ramo aperto saturato} sicuramente non può terminare. Nulla può dirsi in altri casi.

\textbf{Osservazione:} in quali condizioni un \textit{tableaux} in FOL può non essere decidibile? Nel caso in cui la formula contenga un quantificatore universale e il dominio di interpretazione sia infinito. La $\gamma$ rule può essere applicata una volta per ogni costante del dominio (quindi un numero infinito di volte, dato il dominio infinito). Questa condizione è necessaria affinché si produca un tableaux infinito, ma non sufficiente: potrebbe comunque essere possibile applicare la branch closure ad ogni ramo aperto.

\subsection{Open saturated branch}
Un \textit{ramo aperto saturato} (open saturated branch) è un ramo nel quale ogni elemento che non sia un \textit{literal} è stato analizzato almeno una volta e, inoltre, in cui ogni $\gamma$ rule è stata applicata con ogni possibile \textit{term} presente sul ramo.

La costruzione di un \textit{ramo aperto saturato} sicuramente non può terminare. È possibile tentare la costruzione di un modello su un ramo saturato analizzando le formule presenti sul ramo.