\chapter{PL - Introduzione}
La logica delle proposizioni (Propositional Logic -\textbf{PL}) è una logica che permette di rappresentare fatti (affermazioni), che possono essere vere o false.

PL si compone di \textbf{simboli logici} e \textbf{variabili proposizionali}. Una proposizione (formula) è composta quindi di variabili proposizionali uniti da simboli logici. La formula può essere \textit{vera} o \textit{falsa} a seconda dell'assegnazione delle singole variabili.

\begin{fdefinition}[Linguaggio della PL]
\textbf{Logical symbols:}
\begin{enumerate*}[label=(\arabic*)]
\item $\lnot$
\item $\land$
\item $\lor$
\item $\supset$
\item $\equiv$
\end{enumerate*}
\\
\textbf{PL formulas and sub-formulas}
\begin{itemize}
\item every logical variable $P \in \mathrm{P}$ is an atomic formula
\item every atomic formula is a formula
\item if A and B are formulas, then $\lnot A, A \land B, A \lor B, A \supset B, A \equiv B$ are formulas
\end{itemize}
\end{fdefinition}

Una \textbf{funzione di interpretazione} $I: \mathit{P} \to \lbrace \top, \bot \rbrace$ assegna un valore vero o falso a ciascuna variabile $P \in \mathrm{P}$.

Una funzione di interpretazione è detta \textbf{modello} di una funzione $\varphi$ se le sue assegnazioni rendono il valore della funzione vero. In simboli: $I \models \varphi$.

\subsection{SAT, UNSAT, VAL}

\begin{itemize}
\item Una formula $\mathrm{A}$ è soddisfacibile (\textbf{SAT}) se $\exists I$ funzione di interpretazione t.c. $I \models \mathrm{A}$.

\item Una formula $\mathrm{A}$ è insoddisfacibile (\textbf{UNSAT}) se $\nexists I$ funzione di interpretazione t.c. $I \models \mathrm{A}$.

\item Una formula $\mathrm{A}$ è valida (\textbf{VALID}) se $\forall I, I \models \mathrm{A}$
\end{itemize}

\textbf{Osservazione:}

Se $\mathrm{A}$ è \textbf{VALID}, $\lnot \mathrm{A}$ è \textbf{UNSAT}.

Se $\mathrm{A}$ è \textbf{SAT}, $\lnot \mathrm{A}$ non è valida.

Se $\mathrm{A}$ non è valida, $\lnot \mathrm{A}$ è \textbf{SAT}.

Se $\mathrm{A}$ è \textbf{UNSAT}, $\lnot \mathrm{A}$ è \textbf{VALID}. 


\subsection{Conseguenza e equivalenza logica}
\begin{itemize}
\item Una formula $\mathrm{A}$ è una \textbf{conseguenza logica} di un insieme di formule $\Gamma$, in simboli $\Gamma \models \mathrm{A}$ sse per ogni funzione di interpretazione $I$ che soddisfa tutte le formule di $\Gamma$, $I$ soddisfa $\mathrm{A}$.

\item Due formule $\mathrm{A}, \mathrm{B}$ sono \textbf{equivalenti}, in simboli $\mathrm{A} \equiv \mathrm{B}$ sse per ogni funzione di interpretazione $I$, $I(\mathrm{A}) = I(\mathrm{B})$. 
\end{itemize}

\subsection{Procedure di decisione}

\textbf{Model checking ($I$, $\varphi$)}: $I \stackrel{?}{\models} \varphi$ ($I$ soddisfa $\varphi$?)

\textbf{Satisfiability ($\varphi$)}: $\stackrel{?}{\exists} I | I \models \varphi$ (Esiste un modello che soddisfi $\varphi$?)

\textbf{Validity ($\varphi$)}: $\stackrel{?}{\models} \varphi$ ($\varphi$ è soddisfatta da qualsiasi modello?)

\textbf{Logical consequence ($\Gamma$, $\varphi$)}: $\Gamma \stackrel{?}{\models} \varphi$ (Ogni modello che soddisfa $\Gamma$ soddisfa anche $\varphi$?)

\subsection{Proprietà della conseguenza logica}
Siano $\Gamma$ e $\Sigma$ due insiemi di formule proposizionali; $A$, $B$ due formule proposizionali, allora valgono le seguenti proprietà:

\textbf{Reflexivity:} $\lbrace A \rbrace \models A$

\textbf{Monotonicity:} Se $\Gamma \models A$ allora $\Gamma \cup \Sigma \models A$

\textbf{Cut:} Se $\Gamma \models A$ and $\Sigma \cup \lbrace A \rbrace \models B$ allora $\Gamma \cup \Sigma \models B$

\textbf{Compactness:} Se $\Sigma \models A$, allora esiste un sottoinsieme finito $\Gamma_0 \subseteq \Gamma$ tale che $\Gamma_0 \models A$

\textbf{Deduction theorem} Se $\Gamma$, $A \models B$ allora $\Gamma \models A \to B$

\textbf{Refutation principle:} $\Gamma \models A$ se e solo se $\Gamma \cup \lbrace \lnot A \rbrace$ è insoddisfacibile

\subsection{Formalizzazione del linguaggio naturale}

\begin{itemize}
\item $A$: "It is the case that A"
\item $\lnot A$: "It is not the case that A"
\item $A \land B$: "A and B", "A but B", "Although A, B", "Both A and B"
\item $A \lor B$: "A or B", "Either A or B"
\item $A \to B$: "If A, then B", "B if A", "B only if A"
\item $\lnot (A \lor B)$: "Neither A nor B"
\item $\lnot (A \land B)$: "It is not the case that both A and B"
\end{itemize}
