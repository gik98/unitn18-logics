\chapter{PL - Tableaux}

\begin{tabular*}{\textwidth}{c @{\extracolsep{\fill}} c}
$\alpha$ rules & $\lnot \lnot$ elimination \\
\\
%alpha rules
\begin{tabular}{c c c}
\begin{tabular}{c}
$\phi \land \psi$ \\
\hline
$\phi$ \\
$\psi$ \\
\end{tabular} &
\begin{tabular}{c}
$\phi \lor \psi$ \\
\hline
$\lnot \phi$ \\
$\lnot \psi$ \\
\end{tabular} &
\begin{tabular}{c}
$\lnot (\phi \supset \psi)$ \\
\hline
$\phi$ \\
$\lnot \psi$ \\
\end{tabular}
\end{tabular} &
%lnot lnot elimination
\begin{tabular}{c}
$\lnot \lnot \phi$ \\
\hline
$\phi$
\end{tabular}\\
\\
$\beta$ rules & Branch closure \\
%beta rules
\begin{tabular}{c c c}
\begin{tabular}{c}
$\phi \lor \psi$ \\
\hline
$\phi$ \vline \hspace{1mm} $\psi$ 
\end{tabular} &
\begin{tabular}{c}
$\lnot (\phi \land \psi)$ \\
\hline
$\lnot \phi$ \vline \hspace{1mm} $\lnot \psi$ 
\end{tabular} &
\begin{tabular}{c}
$\phi \supset \psi$ \\
\hline
$\lnot \phi$ \vline \hspace{1mm} $\psi$ 
\end{tabular}
\end{tabular} &
%branch closure
\begin{tabular}{c}
$\phi$\\
$\lnot\phi$\\
\hline
$\mathrm{X}$
\end{tabular}\\
\end{tabular*}

L'equivalenza può essere riscritta come doppia implicazione.

$$\phi \equiv \psi \iff (\phi \supset \psi) \land (\psi \supset \phi)$$
