\chapter{ML - Tableaux}

\textit{Il funzionamento del tableaux in ML presenta delle forti analogie con il tableaux della FOL. Sostanzialmente, l'operatore $\Diamond$ di ML può essere mappato sull'operatore $\exists$ di FOL; l'operatore $\Box$ di ML può essere mappato sull'operatore $\forall$ di FOL. Accortezze devono essere prestate sulle relazioni del frame.}
\\

Al solito, il \textbf{tableaux} è un algoritmo per calcolare la soddisfacibilità - insoddisfacibilità - validità di una formula. In logica modale, un tableaux è un albero in cui in ogni nodo è presente un'affermazione relativa alla formula in un certo mondo ($w \models \varphi$, $w \not\models \varphi$), oppure è sfruttata la $R$ relazione per accedere ad un mondo adiacente ($wRw'$), passo necessario per proseguire la costruzione nel caso di formule del tipo $\Box \psi, \Diamond \psi$.

Un ramo è \textbf{chiuso} se contiene la contraddizione$w \models \varphi$ e $w \not \models \varphi$ (analogamente: $w \models \varphi, w \models \lnot \varphi$), altrimenti è detto \textbf{aperto}. Se tutti i rami sono chiusi, il tableaux è detto chiuso.

La logica modale è \textbf{decidibile} (dunque la costruzione del tableaux termina in un numero finito di iterazioni).

\subsection{Regole di espansione per operatori proposizionali}
Sono del tutto analoghe alle classiche $\alpha$ e $\beta$ rules (\textit{alpha equivalence} e \textit{branch splitting}).


\begin{tabular*}{\textwidth}{c @{\extracolsep{\fill}} c}
$\alpha$ rules & $\lnot \lnot$ elimination \\
\\
%alpha rules
\begin{tabular}{c c c}
\begin{tabular}{c}
$w \models \phi \land \psi$ \\
\hline
$w \models \phi$ \\
$w \models \psi$ \\
\end{tabular} &
\begin{tabular}{c}
$w \not \models \phi \lor \psi$ \\
\hline
$w \not \models \phi$ \\
$w \not \models \psi$ \\
\end{tabular} &
\begin{tabular}{c}
$w \not \models \phi \supset \psi$ \\
\hline
$w \models \phi$ \\
$w \not \models \psi$ \\
\end{tabular}
\end{tabular} &
%lnot lnot elimination
\begin{tabular}{c}
$w \not \models \lnot \phi$ \\
\hline
$w \models \phi$
\end{tabular}\\
\\
$\beta$ rules & Branch closure \\
%beta rules
\begin{tabular}{c c c}
\begin{tabular}{c}
$w \models \phi \lor \psi$ \\
\hline
$w \models \phi$ \vline \hspace{1mm} $w \models \psi$ 
\end{tabular} &
\begin{tabular}{c}
$w \not \models \phi \land \psi$ \\
\hline
$w \not \models \phi$ \vline \hspace{1mm} $w \not \models \psi$ 
\end{tabular} &
\begin{tabular}{c}
$w \models \phi \supset \psi$ \\
\hline
$w \not \models \phi$ \vline \hspace{1mm} $w \models \psi$ 
\end{tabular}
\end{tabular} &
%branch closure
\begin{tabular}{c}
$w \models \phi$\\
$w \not \models \phi$\\
\hline
$\mathrm{X}$
\end{tabular}\\
\\
\end{tabular*}

\subsection{Regole di espansione per operatori modali}
\begin{tabular*}{\textwidth}{c @{\extracolsep{\fill}} c}
$\gamma$ rules & $\delta$ rules\\
\begin{tabular}{c c}
%gamma rules
\begin{tabular}{c}
$w \models \Box \varphi$ \\
\hline
$wRw'$\\
$w' \models \varphi$
\end{tabular} &
\begin{tabular}{c}
$w' \not \models \Diamond \varphi$ \\
\hline
$wRw'$\\
$w' \not \models \varphi$
\end{tabular}
\end{tabular} &
\begin{tabular}{c c}
%delta rules
\begin{tabular}{c}
$w \not \models \Box \varphi$ \\
\hline
$wRw'$\\
$w' \not \models \varphi$
\end{tabular} &
\begin{tabular}{c}
$w \models \Diamond \varphi$ \\
\hline
$wRw'$ \\
$w' \models \varphi$
\end{tabular}
\end{tabular}\\
\\
\end{tabular*}

\begin{enumerate}
\item Nelle $\gamma$ rules, $w'$ rappresenta un qualsiasi mondo, anche già presente nel tableaux.
\item Nelle $\delta$ rules, $w'$ rappresenta un nuovo mondo non presente nel tableaux.
\end{enumerate}
